\documentclass{article}
\usepackage[utf8]{inputenc}
\usepackage{amsmath}
\begin{document}
\section{Correction of Currents}
Problem: The pad and guard-ring currents could not be measured separately with the sensor mounted on the PCB. How can one calculate the pad current from the total current?
\\ \\
The pad and total current were measured separately only after 80min of annealing after irradiation, before the sensor was mounted on the PCB for TCT measurements. Only measurements of the total current could be performed for the following annealing steps.\\
The aim is to investigate the annealing behavior of the pad current of an LGAD by comparison to the Hamburg model.\\
The sought after quantity is the current related damage rate resulting from the pad current
\begin{equation}
\alpha(t)=\frac{I^{pad}(t)}{\Phi V^{pad}}.
\end{equation}
The sensors were irradiated with fluence $\Phi$. For an LGAD $\alpha(t)$ should differ from the literature value $\alpha^{hm}(t)$ by the multiplication factor $M$.\\
The used variables are
\begin{itemize}
\item $I^{tot}_{80}$, $I^{pad}_{80}$, $I^{gr}_{80}$: Total, pad and guard ring current measured after 80 minutes of annealing on the probe station
\item $I^{tot}(t)$, $I^{pad}(t)$, $I^{gr}(t)$: Total, pad and guard ring current measured after consecutive annealing. Only $I^{tot}(t)$ is known. $I^{pad}(t)$ and $I^{gr}(t)$ are unknown.
\item $V^{tot}$, $V^{pad}$, $V^{gr}$: Total, pad and guard ring volume contributing to leakage current
\item $M_{80}$: The multiplication factor after irradiation and 80min annealing, e.g. from TCT measurements.
\end{itemize}
\subsection{Approach I: Scaling by a constant factor}
Define a constant factor from the current measurement after 80min
\begin{equation}\label{eq:F}
F=\frac{I^{tot}_{80}}{I^{pad}_{80}}
\end{equation}
and calculate the current related damage rate
\begin{equation}\label{eq:alphatilde}
\tilde{\alpha}(t)=\frac{I^{tot}(t)}{F \Phi V^{pad}}.
\end{equation}
The underlying assumption is that the ratio of pad to total current stays the same during annealing, which is not the case, if the gain layer behaves differently than the bulk.
\subsection{Approach II: Predict annealing of the guard ring current}
First assumption: Pad and guard ring currents are generated in the silicon bulk in distinct volumes and only the pad current undergoes multiplication such that
\begin{equation}\label{eq:firstass}
\frac{I^{pad}_{80}}{M_{80}I^{gr}_{80}}=\frac{V^{pad}}{V^{gr}}.
\end{equation}
Second assumption: The guard ring current follows
\begin{equation}\label{eq:secass}
I^{gr}(t)=\alpha^{hm}(t)\Phi V^{gr}.
\end{equation}
Using eq. \ref{eq:F}, \ref{eq:firstass} and \ref{eq:secass} one can write
\begin{align}
\alpha^{\ast}(t)&=\frac{I^{pad}(t)}{\Phi V^{pad}}=\frac{I^{tot}(t)-I^{gr}(t)}{\Phi V^{pad}}\\
&=\frac{I^{tot}(t)}{\Phi V^{pad}}-\alpha^{hm}(t)M_{80}(F-1) \label{eq:alphastar}
\end{align}
The underlying assumption here is that the volumes which generate the currents $I^{pad}$ and $I^{gr}$ stay the same but the current ratio can change, e.g. due to different annealing behavior. $F$ appears in the equation to calculate the volume $V^{gr}$ based on the measurements at 80min annealing at which also $F$ is calculated.
\subsection{Difference between approaches}
The difference between eq. \ref{eq:alphatilde} and \ref{eq:alphastar} is
\begin{equation}
\alpha^{\ast}(t)-\tilde{\alpha}(t)=\underbrace{(F-1)}_{\geq 0}\bigg [\underbrace{\frac{I^{tot}(t)}{F \Phi V^{pad}}}_{=\tilde{\alpha}(t)}-M_{80}\alpha^{hm}(t)\bigg ]
\end{equation}
Therefore $\alpha^{\ast}(t)-\tilde{\alpha}(t)=0$, if $F=1$ (no guard ring current) or if ${\tilde{\alpha}(t)=M_{80}\alpha^{hm}(t)}$, what is to show.

\end{document}